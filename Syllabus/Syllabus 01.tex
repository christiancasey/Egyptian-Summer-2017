\documentclass[11pt]{article}

\usepackage[utf8]{inputenc}
\usepackage[margin=2cm]{geometry}
\usepackage{cite}
\usepackage{natbib}
\usepackage{bibentry}
\usepackage{hyperref}
\usepackage{outlines}
\usepackage{enumitem}
\setenumerate[1]{label=\Roman*.}
\setenumerate[2]{label=\Alph*.}
\setenumerate[3]{label=\roman*.}
\setenumerate[4]{label=\alph*.}

\nobibliography*
	
\begin{document}
	
	\title{An Introduction to Egyptian Hieroglyphs}
	\author{C. Casey \\
		Wilbour Hall 302 \\
  		\texttt{\href{mailto:christian_casey@brown.edu}{christian\_casey@brown.edu}}}
	\date{\today}
	\maketitle
	
	\section*{Class Meeting Time/Place}
	
		\begin{tabular}{l l}
		Date: & July 17 – July 21, 2017 \\
		Time: & 15:30 – 18:20 \\
		Place: & TBD
		\end{tabular} 
		
% Description		
	\section*{Course Description}
		Egyptian has the longest written history of any language, with surviving texts spanning four millennia, from approximately 3200 BC to at least 1100 AD. However, knowledge of this ancient language was lost until the discovery of the Rosetta Stone in 1799 AD. There are no longer any native speakers of Egyptian, and the language can only be studied through an imperfectly-understood script. Despite these shortcomings, Egyptologists have made great headway in understanding this once incomprehensible script and the language it ultimately represents. 
	
		One of the greatest barriers to learning the hieroglyphs is their lack of connection to a spoken language. Egyptologists generally work with an artificial reconstruction of the language which bears little similarity to actual spoken Egyptian. This scholarly form of Egyptian is difficult to understand for new students, especially those who lack experience with advanced studies of philology, phonetics, and syntax. However, thanks to new information about pronunciation and grammar preserved in later Coptic texts, which is still being developed through ongoing research, it is possible to teach Egyptian with a deliberate focus on what we do know about the spoken language. 
	
		In keeping with this aim, this course will teach the most recent stage of Egyptian that was written using the hieroglyphic script: Late Egyptian. This method will enable students to learn the hieroglyphic script through the medium of a language that they can practice by speaking aloud. The goal behind this approach, which is supported by extensive scientific research in the field of second-language acquisition, is to create an introduction to Egyptian which provides the best foundation for long-term study, and which communicates the crucial message that Hieroglyphic script represents a real language, which was once spoken by real people who lived lives that were not entirely different from ours. 
			
		
		
	\section*{Course Objectives}
	
		By the end of the course, students will be able to read short Egyptian texts similar to those studied in class. They will have a broad understanding of the history of the Egyptian language, and they will be well prepared to begin a serious study of Egyptian at the undergraduate level. Students will also develop a familiarity with the resources available to them so that they can continue their studies on their own if they wish.
	
	
	\section*{Course Requirements}
	
	\subsection*{Prerequisites}
		
		This course requires some knowledge of English grammar so that parallel concepts in Egyptian can be introduced quickly. 
		Students who have not studied grammar in school (at an advanced level) will be expected to read a book on English grammar in preparation for the course.
		
		
		The following diagnostic questions will help you determine whether you need to brush up on English grammar:
		\begin{itemize}
			\item What is the difference between the definite and indefinite article?
			\item Which type of article (definite or indefinite) shows the most similarity to the demonstrative adjectives? Can you come up with a hypothetical explanation for this?
			\item What differentiates an interrogative pronoun from a relative pronoun? In your opinion, why do they look so similar?
			\item What is the difference between a gerund and a participle used substantively? What do their definitions have in common?
		\end{itemize}
		
		If all of the terms in these questions are readily familiar, and if you can come up with an answer (even if you're still a little unsure about your answer), you can  expect to understand all of the concepts necessary for learning Late Egyptian.
		If any of these questions is impossible for you to answer, 
			don't despair, nothing here is especially difficult once you've seen an explanation, 
			but you will need to study English grammar before the first day of class.
		Consider using one of the \href{https://www.amazon.com/s/ref=nb_sb_noss_2?url=search-alias\%3Dstripbooks&field-keywords=english+grammar+for+students}{\emph{English Grammar for Students of X} books}, where \emph{X} is a language that you want to learn more about
			(e.g. \href{https://www.amazon.com/English-Grammar-Students-Latin-Learning/dp/0934034346/ref=sr_1_1?s=books&ie=UTF8&qid=1490732513&sr=1-1&keywords=english+grammar+for+students+of+latin}{\emph{English Grammar for Students of Latin}}).
		These books offer very precise definitions of grammatical terms that are useful for the study of foreign languages, 
			but they also contain a lot of material on the second language to be studied,
			so choose a second language that you are also interested in studying
			(N.B. An advanced degree in Egyptology will require reading proficiency in French and German, so those are good languages to start learning now).
			If you're feeling particularly adventurous, consider reading \href{https://www.amazon.com/English-Grammar-Students-Arabic-Learning/dp/0934034354}{\emph{English Grammar for Students of Arabic}}.
			
	\subsection*{Materials}
		In addition to the usual basic school supplies (notebook and pencils), students will also be required to bring index cards ($\approx 200$).
		These will be used to make flashcards for studying signs and vocabulary outside of class.
		
% Coursework	
	\section*{Coursework}
		\subsection*{Attendance \& Participation}
			Attending class and participating in group activities are mandatory and represent a significant portion of your total evaluation for this course.
			Some of the in-class activities are designed to encourage collaboration with other students,
			and students will be expected to work together.
%			They are also designed to be fun and challenging, so you should have no trouble working together and getting all 20 points.
		
		\subsection*{Homework}
			Each day, beginning with the first class, students will be asked to complete a homework assignment to practice a crucial skill learned that day.
				This is necessary because subsequent classes assume a thorough knowledge of the previous day's most important topics.
			There will be four homeworks in total (Monday-Thursday).
			
		\subsection*{Quizzes}
			Beginning with the second day of class, there will be a quiz to evaluate the success of the previous night's homework assignment 
			and/or encourage students to learn the vocabulary.
			The quizzes will be largely diagnostic, but they will still be graded and those grades included in the final score for the course.
			There will be four quizzes in total (Tuesday-Friday).
			
		\subsection*{Final Projects}
			There will be two final projects in this course.
			The first, to be completed in class on Friday, will be the translation of an unseen Egyptian text.
			The second, which we will start working on before the last day of class, 
			will be to compose a new text in ancient Egyptian
			 and to write it on papyrus like a real Egyptian scribe. 
			Students will be allowed to take the second project home as a souvenir.
			
			
		\subsection*{Grading}
			\begin{tabular}{l l}
				Attendance \& Participation & 20\% \\
				Homework & 20\% \\
				Quizes & 20\% \\
				Final Projects & 40\% \\
			\end{tabular}
	
	\section*{Course Calendar}
	
	(N.B. This is only an outline. 
	A complete syllabus will be handed out on the first day of class.)

	\subsection*{Monday, July 17}
		\subsubsection*{Class 1 -- History Lessons}
			\begin{outline}[itemize]
				\1 The History of Egyptology
					\2 Decipherment Challenge 1 (Thomas Young)
						\3 Using royal names from the Rosetta Stone (and elsewhere), 
							decipher some basic Egyptian phonograms, 
							just as Thomas Young did in the early 1800s.
				
				\1 The History of Egyptian Languages
					\2 Egyptian Arabic
						\3 In-class activity: Decipher modern Egyptian Arabic
					\2 Coptic
						\3 Homework 1: Reading Coptic
					\2 Demotic
						\3 Brief glance at the instructor's dissertation research
					\2 Hieroglyphic
						\3 Historical overview and introduction to the main subject of this class
			\end{outline}
			
	\subsection*{Tuesday, July 18}
		\subsubsection*{Class 2 -- The Hieroglyphic Script}
			\begin{outline}[itemize]
				\1 Review of the previous day's lesson
					\2 Hand-in Homework 1
					\2 Quiz 1 (Scripts)
				\1 Reading Hieroglyphs
					\2 Basics (Part 1)
						\3 In-class activity: Determining glyph sequence
					\2 Decipherment Challenge 2 (J.-F. Champollion)
						\3 Using examples of corresponding Coptic and Hieroglyphic words (and maybe a bit of help from the instructor), 
							decipher the Egyptian script, 
							just as Jean-Fran\c{c}ois Champollion did in 1822.
					\2 Basics (Part 2)
						\3 Correct our initial decipherments
						\3 Learn about the three types of hieroglyphs
						\3 Learn the current Egyptological transcription system
					\2 Script Materiality and Media
						\3 Learn about carved hieroglyphs
						\3 Learn about papyrus and cursive scripts
				\1 Writing Hieroglyphs
					\2 Egyptian calligraphy
						\3 Homework 2: Practice cursive hieroglyphic signs
					\2 Studying vocabulary
						\3 In-class activity: Creating Egyptian flashcards for independent study
				
				\1 Grammar (Part 1)
					\2 Overview of the Egyptian Parts of Speech
					\2 Nouns
			\end{outline}
			
	\subsection*{Wednesday, July 19}
		\subsubsection*{Class 3 -- Building Blocks}
			\begin{outline}[itemize]
				\1 Review of the previous day's lesson
					\2 Hand-in Homework 2
					\2 Quiz 2 (Homework \& Vocabulary)
				
				\1 Grammar (Part 2)
					\2 Articles
					\2 Prepositions
					\2 Pronouns 
						\3 Homework 3: Pronouns and their combinations
						\3 In-class activity: Reproduce simple Egyptian phrases
					\2 Adverbs
					\2 Quantifiers and Copulas
						\3 In-class activity: Translate sentences expressing existence and possession
			\end{outline}
			
	\subsection*{Thursday, July 20}
		\subsubsection*{Class 4 -- Connections}
			\begin{outline}[itemize]
				\1 Review of the previous day's lesson
					\2 Hand-in Homework 3
					\2 Quiz 3 (Vocabulary)
				\1 Grammar (Part 3)
					\2 Adjectives
						\3 In-class activity: Create descriptive sentences in Egyptian
					\2 Verbs (Basic)
						\3 In-class activity: Practicing simple verb conjugations
						\3 Homework 4: Translation of authentic Egyptian sentences
			\end{outline}
			
	\subsection*{Friday, July 21}
		\subsubsection*{Class 5 -- Putting it all together}
			\begin{outline}[itemize]
				\1 Review of the previous day's lesson
					\2 Hand-in Homework 4
					\2 Quiz 4 (Vocabulary)
				\1 Grammar (Part 4)
					\2 Verbs (Advanced)
						\3 In-class activity: Exploring the complexities of the Egyptian verb
				\1 Final Projects
					\2 Translation 
						\3 Translate an unseen Egyptian text into English
					\2 Composition
						\3 Compose a simple text in Egyptian
						\3 Write the text using cursive hieroglyphs on papyrus
					
			\end{outline}
	
	%\nocite{*}
	%
%	\nobibliography{/Users/christiancasey/Dropbox/BibTeX/3rd-yearProjects-Syllabus}
%	\bibliographystyle{apalike}



\end{document}