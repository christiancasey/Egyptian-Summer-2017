\documentclass[11pt]{article}

\usepackage[utf8]{inputenc}
\usepackage[margin=2cm]{geometry}
\usepackage{cite}
\usepackage{natbib}
\usepackage{bibentry}
\usepackage{hyperref}
\usepackage{outlines}
\usepackage{enumitem}
\setenumerate[1]{label=\Roman*.}
\setenumerate[2]{label=\Alph*.}
\setenumerate[3]{label=\roman*.}
\setenumerate[4]{label=\alph*.}

\nobibliography*
	
\begin{document}
	
	\title{An Introduction to Egyptian Hieroglyphs}
	\author{C. Casey \\
		Wilbour Hall 302 \\
  		\texttt{\href{mailto:christian_casey@brown.edu}{christian\_casey@brown.edu}}}
	\date{\today}
	\maketitle
	
	\section*{Class Meeting Time/Place}
	
		\begin{tabular}{l l}
		Date: & July 17 -- July 21, 2017 \\
		Time: & 15:30 -- 18:20 \\		% Change to 9-12
		Place: & J. Walter Wilson 303		% Change to 403
		\end{tabular}
		
	\section*{Office Hours}
		\begin{tabular}{l l}
		Date: & July 17 -- July 21, 2017 \\
		Time: & 13:30 -- 15:00 \\					% Change to 12-1:30
		Place: & Wilbour Hall 302
		\end{tabular}
		
% Description		
	\section*{Course Description}
		Egyptian has the longest written history of any language, with surviving texts spanning four millennia, from approximately 3200 BC to at least 1100 AD. However, knowledge of this ancient language was lost until the discovery of the Rosetta Stone in 1799 AD. There are no longer any native speakers of Egyptian, and the language can only be studied through an imperfectly-understood script. Despite these shortcomings, Egyptologists have made great headway in understanding this once incomprehensible script and the language it ultimately represents. 
	
		One of the greatest barriers to learning the hieroglyphs is their lack of connection to a spoken language. Egyptologists generally work with an artificial reconstruction of the language which bears little similarity to actual spoken Egyptian. This scholarly form of Egyptian is difficult to understand for new students, especially those who lack experience with advanced studies of philology, phonetics, and syntax. However, thanks to new information about pronunciation and grammar preserved in later Coptic texts, which is still being developed through ongoing research, it is possible to teach Egyptian with a deliberate focus on what we do know about the spoken language. 
	
		In keeping with this aim, this course will teach the most recent stage of Egyptian that was written using the hieroglyphic script: Late Egyptian. This method will enable students to learn the hieroglyphic script through the medium of a language that they can practice by speaking aloud. The goal behind this approach, which is supported by extensive scientific research in the field of second-language acquisition, is to create an introduction to Egyptian which provides the best foundation for long-term study, and which communicates the crucial message that Hieroglyphic script represents a real language, which was once spoken by real people who lived lives that were not entirely different from ours. 
			
		
		
	\section*{Course Objectives}
	
		By the end of the course, students will be able to read short Egyptian texts similar to those studied in class. They will have a broad understanding of the history of the Egyptian language, and they will be well prepared to begin a serious study of Egyptian at the undergraduate level. Students will also develop a familiarity with the resources available to them so that they can continue their studies on their own if they wish.
	
	
	\section*{Course Requirements}
	
	\subsection*{Prerequisites}
		
		This course requires some knowledge of English grammar so that parallel concepts in Egyptian can be introduced quickly. 
		Students who have not studied grammar in school (at an advanced level) will be expected to read a book on English grammar in preparation for the course.
		
		
		The following diagnostic questions will help you determine whether you need to brush up on English grammar:
		\begin{itemize}
			\item What is the difference between the definite and indefinite article?
			\item Which type of article (definite or indefinite) shows the most similarity to the demonstrative adjectives? Can you come up with a hypothetical explanation for this?
			\item What differentiates an interrogative pronoun from a relative pronoun? In your opinion, why do they look so similar?
			\item What is the difference between a gerund and a participle used substantively? What do their definitions have in common?
		\end{itemize}
		
		If all of the terms in these questions are readily familiar, and if you can come up with an answer (even if you're still a little unsure about your answer), you can  expect to understand all of the concepts necessary for learning Late Egyptian.
		If any of these questions is impossible for you to answer, 
			don't despair, nothing here is especially difficult once you've seen an explanation, 
			but you will need to study English grammar before the first day of class.
		Consider using one of the \href{https://www.amazon.com/s/ref=nb_sb_noss_2?url=search-alias\%3Dstripbooks&field-keywords=english+grammar+for+students}{\emph{English Grammar for Students of X} books}, where \emph{X} is a language that you want to learn more about
			(e.g. \href{https://www.amazon.com/English-Grammar-Students-Latin-Learning/dp/0934034346/ref=sr_1_1?s=books&ie=UTF8&qid=1490732513&sr=1-1&keywords=english+grammar+for+students+of+latin}{\emph{English Grammar for Students of Latin}}).
		These books offer very precise definitions of grammatical terms that are useful for the study of foreign languages, 
			but they also contain a lot of material on the second language to be studied,
			so choose a second language that you are also interested in studying
			(N.B. An advanced degree in Egyptology will require reading proficiency in French and German, so those are good languages to start learning now).
			If you're feeling particularly adventurous, consider reading \href{https://www.amazon.com/English-Grammar-Students-Arabic-Learning/dp/0934034354}{\emph{English Grammar for Students of Arabic}}.
			
	\subsection*{Assignments and Readings}
		Most assignments and readings will be handed out in hard copy in class
			(except Extra Practice assignments, which are optional).
		All will be posted on the Canvas website so that they can be accessed electronically at any time.
		Quizzes and exams will not be on Canvas, but they will be made available online after the end of the course.
	
	\subsection*{Materials}
		In addition to the usual basic school supplies (notebook and pencils), students will also be required to bring index cards ($\approx 200$).
		These will be used to make flashcards for studying signs and vocabulary outside of class.
		
		
% Coursework	
	\section*{Coursework}
		\subsection*{Attendance \& Participation}
			Attending class and participating in group activities are mandatory and represent a significant portion of your total evaluation for this course.
			Some of the in-class activities are designed to encourage collaboration with other students,
				and students will be expected to work together.
		
		\subsection*{Homework}
			Each day, beginning with the first class, students will be asked to complete a homework assignment to practice a crucial skill learned that day.
				This is necessary because subsequent classes assume a thorough knowledge of the previous day's most important topics.
				Most of these assignments are meant to teach vocabulary, which will be tested on the following day during the quiz.
			There will be four homeworks in total (Monday-Thursday).
			
		\subsection*{Quizzes}
			Beginning with the second day of class, there will be daily quizzes on vocabulary and grammar.
			The quizzes will be largely diagnostic, but they will still be graded and those grades included in the final score for the course;
				however, they will not be graded as strictly as other assignments.
			The quiz scores will be curved up so that the lowest possible quiz grade will be a 70, 
				meaning that everyone who takes the quiz is guaranteed at least a passing grade for that part of the course.
			There will be four quizzes in total (Tuesday-Friday).
			
		\subsection*{Final Exam}
			On Friday, students will take a short final exam covering the material from the literature reading.
			Like the quizzes, this exam will be graded on a curve so that no score is below passing.
			
		\subsection*{Final Project}
			For their final project, students will compose an original text in Ancient Egyptian
			 and write it on papyrus like a real Egyptian scribe. 
			Students will be allowed to take the second project home as a souvenir.
			
			
		\subsection*{Grading}
			\begin{tabular}{l l}
				Attendance \& Participation & 20\% \\
				Homework & 20\% \\
				Quizes & 10\% \\
				Final Exam & 25\% \\
				Final Project & 25\% \\
			\end{tabular}
	
	\section*{Course Calendar}
	
	Each day is divided into four parts: 
	\begin{enumerate}
	\item In-Class Activities
	\item Homework
	\item Reading
	\item Extra Practice
	\end{enumerate}
	
	Extra practice is a way for students to study beyond the course material.
	It is always optional.
	Extra practice materials will not be discussed during class, but interested students are welcome to discuss them with me during office hours.
	
	\subsection*{Monday, July 17}
		\subsubsection*{Class 1 -- I. In-Class Activities}
			\begin{outline}[itemize]
				\1 Rosetta Stone
					\2 Using royal names from the Rosetta Stone (and elsewhere), 
						decipher some basic Egyptian phonograms, 
						just as Thomas Young did in the early 1800s.
				
				\1 Nametags, Meet your Classmates
					\2 Using the uniliteral signs learned in the previous activity, make a nametag for yourself in the hieroglyphic script.
						Use these nametags to learn your classmates' names and introduce yourself.
				\1 Sign Types
					\2 Learn the other types of hieroglyphic signs: Multiliterals and Determinatives.

%%%%%%%%%%%%%%%%%%%%%%%%%%%%%%%%%%%%			
				\1 The Begatitudes
					\2 Using the names found in the Genealogy of Jesus from Matthew, decipher the Coptic script.
						We will use Coptic throughout the rest of the class to learn the pronunciations of Egyptian words.
						
			\end{outline}
			
		\subsubsection*{Class 1 -- II. Homework}
			Label the objects around your house with the household object stickers.
			Then make flashcards of these words and study them.
			Identify any new signs that you have not learned yet, and try to determine their meanings and/or phonetic values.
			
		\subsubsection*{Class 1 -- III. Reading}
			(Allen, 2014, pp. 1-12) -- The history of the Egyptian language.
			
		\subsubsection*{Class 1 -- IV. Extra Practice (Optional)}
			(Gardiner, 1957, pp. 442-548) -- Study the signs and what they represent.
			
	\subsection*{Tuesday, July 18}
		\subsubsection*{Class 2 -- I. In-Class Activities}
			\begin{outline}[itemize]
				\1 Quiz -- Uniliterals
				\1 Quiz -- Household Objects
				
				\1 Review Homework
					\2 Discuss new signs and their meanings or phonetic values.
				
				\1 New Sign Calligraphy
					\2 Learn to write our newly-learned hieroglyphs as an Egyptian scribe would have.
				
				\1 Create a Timeline of the Egyptian languages
					\2 Design a timeline that shows the development of Egyptian over its long history
						based on what you learned in last night's reading.
					\2 In groups, combine your individual ideas and make a huge timeline to hang on the wall.
				
				\1 Vocabulary
					\2 Learn some basic Egyptian vocabulary -- nouns, articles, and pronouns -- in preparation for the following grammar activity.
					
				\1 Possessive articles and nouns
					\2 Practice creating possessive noun phrases using the printed cards.
					\2 Test your partner by inventing new combinations.
					
			\end{outline}
			
		\subsubsection*{Class 2 -- II. Homework}
			Make flashcards of the new vocabulary you were given at the end of class (Nouns, Pronouns).
			Identify any new signs that you have not learned yet, and try to determine their meanings and/or phonetic values.
			Read \emph{The Misadventures of Wenamun} comic book.
			
		\subsubsection*{Class 2 -- III. Reading}
			(Manley, 1996, pp. 70-73, 94-95, 98-99) -- Egyptian geography during the Third Intermediate Period.
			
		\subsubsection*{Class 2 -- IV. Extra Practice (Optional)}
			(Goedicke, 1975, pp. 149-158) -- The complete story of Wenamun in translation.\\
			(Depuydt, 1993, pp. 1-5) -- A paper by a Brown professor on Egyptian pedagogy, 
				which inspired the methodology of this week's grammar lessons.
			
	\subsection*{Wednesday, July 19}
		\subsubsection*{Class 3 -- I. In-Class Activities}
			\begin{outline}[itemize]
				\1 Quiz -- Nouns and Pronouns
				\1 Quiz -- Noun Phrases
				\1 Quiz -- Wenamun Story
				
				\1 Review Homework
					\2 Discuss new signs and their meanings or phonetic values.
				
				\1 New Sign Calligraphy
					\2 Learn to write our newly-learned hieroglyphs as an Egyptian scribe would have.
					
				\1 Create a map of Ancient Egypt and the Eastern Mediterranean
					\2 Using the \emph{Atlas of Ancient Egypt} (part of which you read for homework)
						locate places from the list of geographical nouns on your own map.
					\2 In groups, use your individual maps to create a large, poster-sized map to hang on the wall.
				
				\1 Prepositions and prepositional phrases
					\2 Practice creating prepositional phrases using the printed cards (both from today and yesterday).
					\2 Test your partner by inventing new combinations.
					
			\end{outline}
			
		\subsubsection*{Class 3 -- II. Homework}
			Make flashcards of the new vocabulary you were given at the end of class (Verbs, Prepositions).
			Study them diligently, as you will need to know these words for tomorrow's grammar lesson.
			Identify any new signs that you have not learned yet, and try to determine their meanings and/or phonetic values.
			
		\subsubsection*{Class 3 -- III. Reading}
			(Gardiner, 1957, pp. 5-11) -- The history of the various Egyptian scripts.
			
		\subsubsection*{Class 3 -- IV. Extra Practice (Optional)}
			(Casey, 2008, pp. 88-100) -- Learn Hieratic using the sign list that I created as part of my undergraduate thesis.
			
	\subsection*{Thursday, July 20}
		\subsubsection*{Class 4 -- I. In-Class Activities}
			\begin{outline}[itemize]
				\1 Quiz -- Verbs
				\1 Quiz -- Prepositional Phrases
				\1 Quiz -- Nominal Sentences
				
				\1 Visit the Hay Library
					\2 We will take a short field trip to Brown's own Hay library, 
						where we will see real examples of Egyptian papyri.
				
				\1 Review Homework
					\2 Discuss new signs and their meanings or phonetic values.
				
				\1 New Sign Calligraphy
					\2 Learn to write our newly-learned hieroglyphs as an Egyptian scribe would have.
					
				\1 Verbs
					\2 Discuss the different types of verbal sentences in Egyptian.
					\2 Translate examples together as a class.
					\2 Create your own examples in small groups.
				
				\1 Particles, Adjectives, Adverbs
					\2 Introduce the remaining (and comparatively rare) parts of speech.
					\2 Translate examples as a class.
					
				\1 Storytime
					\2 Read the Wenamun story together for the remaining class time.
				
				
			\end{outline}
			
		\subsubsection*{Class 4 -- II. Homework}
			Make flashcards of the new vocabulary you were given at the end of class (Particles, Adjectives, Adverbs).
			Identify any new signs that you have not learned yet, and try to determine their meanings and/or phonetic values.
			
		\subsubsection*{Class 4 -- III. Reading}
			Study the Wenamun story with transcription and translation.
			Understand the parts you can understand, mark the parts you cannot for discussion in class tomorrow.
			
		\subsubsection*{Class 4 -- IV. Extra Practice (Optional)}
			Try your hand at the unabridged, original text of the Wenamun story,
				complete with variant spellings, uncertain readings, and large patches of missing text (lacunae).
			Use the complete translation from Goedicke, 1975 to help you.
			
	\subsection*{Friday, July 21}
		\subsubsection*{Class 5 -- I. In-Class Activities}
			\begin{outline}[itemize]
				\1 Exam -- Literature
				\1 Quiz -- All Vocabulary
				
				\1 Final Project -- Create your own Egyptian text
					\2 Prepare composition in English and discuss with teacher.
					\2 Learn vocabulary needed to produce Egyptian text.
					\2 Identify any new hieroglyphs in the vocabulary and learn to draw them.
					\2 Work out grammar of text and implement Egyptian version.
					\2 Draw text with cursive hieroglyphs on papyrus in the style of an Egyptian funerary text.
					\2 Read your text aloud to your classmates (optional).
				
					
			\end{outline}
	
	%\nocite{*}
	%
%	\nobibliography{/Users/christiancasey/Dropbox/BibTeX/3rd-yearProjects-Syllabus}
%	\bibliographystyle{apalike}



\end{document}